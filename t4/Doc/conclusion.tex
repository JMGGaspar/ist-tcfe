\section{Conclusion}
\label{sec:conclusion}

\indent

In this laboratory assignment, the main objectives were achieved: we were able to analyse the working principle of an Audio Amplifier, as well as understanding the various circuit architectures required to solve this problem. We were also able to work with circuits containing transistors, including this component for the first time in our work. 

The results from both sources, \textit{Octave} and \textit{ngspice}, matched in their general form, overall. However, some significant differences exist. These can be explained due to different computational models being used in each program. While \textit{ngspice} works as a simulation, with a realistic model, our \textit{Octave} computations used a theoretical DC model, which has imperfections and uses some approximations that  introduce discrepancies in the results. The theoretical model used is not completely appropriate and more work can be conduced in this aspect.


\subsection{Further Work}
\indent 

Due to time constraints, this lab assignment was knowingly and deliberately less complete than usual. This fact reflects on the precision and on the matching of theoretical and simulated results. 

Further work could be developed in perfecting the {\it Octave} theoretical model so as to make it more refined and to make it match the simulated (realistic) model.
