\section{Introduction}
\label{sec:introduction}

% state the learning objective 

\indent

The objective of this laboratory assignment is to create a circuit that works as an {\bf audio amplifier}. 

An audio amplifier is a device that aims to amplify the intensity of a sound, given as an input, while avoiding the introduction of noise or sound deformation. 

This circuit does not process sound waves. It receives as an input an analog electrical signal, which is a processed sound wave (for example, by a microphone), and amplifies it, producing an amplified analog electrical signal as an output. 

The main challenge is to find a good balance between cost of production, width of the pass-band, and gain, i.e. a non-expensive amplifier that is able to increase the sound amplitude without significant losses in higher and lower frequencies. 

The circuit, which can be seen on the following figure (Figure \ref{fig:schematic}), is composed by two stages: the \textit{gain stage} and the \textit{output stage}. 
These two stages work in complement to each other: the \textit{gain stage}, which aims to increase the magnitude of the signal, without much concern for the output impedance, while the \textit{output stage}, aims to convert the output impedance into something acceptable for the desired load (in this case it is $8 k\Omega$), without affecting the previous gain.


\begin{figure}[h!]
    \centering
    \includegraphics[width = 0.8\linewidth]{fig0.pdf}
    \caption{Audio amplifier circuit}
    \label{fig:schematic}
\end{figure}

In Section~\ref{sec:analysis}, a theoretical analysis of the circuit is
presented. In Section~\ref{sec:simulation}, the circuit is analysed by
simulation, and the results are compared to the theoretical results obtained in Section~\ref{sec:analysis}. The conclusions of this study are outlined in Section~\ref{sec:conclusion}.
