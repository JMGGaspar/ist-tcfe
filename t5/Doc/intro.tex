\section{Introduction}
\label{sec:introduction}

% state the learning objective 

\indent

The objective of this laboratory assignment is to develop a \textbf{Band-Pass filter (BPF)} circuit.

A Band-Pass filter is, as its name indicates, an analogue filter through which, ideally, only frequencies inside a certain range can pass without being affected. This range is called the \textit{pass-band} of the filter, and is bounded by both a higher and a lower cutoff frequencies. Outside of the pass-band, the frequencies are rejected (attenuated). 

In our case, the Band-Pass filter is a combination of an amplifier stage with two other filters: a High-Pass filter (HPF) and a Low-Pass filter (LPF), which are associated with the lower and higher cutoff frequencies, respectively. 

In this assignment, the main challenge is to choose an adequate circuit architecture and to optimise the parameters to achieve the desired specifications: a 40 $dB$ gain with the pass-band centred at 1 $kHz$.  

We were introduced to the Op-Amp component and its functionalities, and included it for the first time in our work. 



\begin{figure}[h!]
    \centering
    \includegraphics[width = 0.9\linewidth]{fig2.pdf}
    \caption{Band pass filter circuit (with amplifier)}
    \label{fig:schematic}
\end{figure}
