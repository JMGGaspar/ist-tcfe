\section{Conclusion}
\label{sec:conclusion}

\indent

In this laboratory assignment, the main objectives were achieved: we were able to analyse the working principle of an AC/DC converter, as well as understanding the various circuit architectures to solve this problem. We were also able to work with circuits containing diodes. 

In order to analyse the circuit, \textit{Octave} was used to compute the theoretical values and \textit{Ngspice} was used to simulate the circuit shown in Figure \ref{fig:Circuit}. To obtain the theoretical values, Kirchhoff Laws were used as well as the $V_{on}$ model for the diode values.
After that, a plot was made with the values of $V_o$ (voltage output) over time, obtaining a near horizontal line (DC) at 12 volts. 

With the theoretical values in, the circuit was simulated and the same plot was made, showing similar results. The input values were then optimised in order to achieve the best simulation output possible. With this, the objective of the laboratory was achieved and we were able to successfully build an AC/DC converter.
